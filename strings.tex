%%%%%%%%%%%%%%%%%%%%%%%%%%%%%%%%%%%%%%%%%%%%%%%%%%%%%%%%%%%%%%%%%%%%%%%%
\chapter{Data Structures for Strings}\chaplabel{strings}
\bibliographystyle{plain}

In this chapter, we consider data structures for storing strings;
sequences of characters taken from some alphabet.  Data structures for
strings are an important part of any system that does text processing,
whether it be a text-editor, word-processor, or Perl interpreter.

Formally, we study data structures for storing sequences of symbols
over the alphabet $\{-1,\ldots,N-1\}$.  We assume that all strings are
terminated with the special character $\$=-1$ and that $\$$ only
appears as the last character of any string.  In real applications,
strings are not usually over the alphabet $\{-1,\ldots,N-1\}$.
Normally they would be ASCII or Unicode strings.  However, the ASCII
and Unicode alphabets can easily be translated into an alphabet of 
the form $\{-1,\ldots,N-1\}$.

Storing strings of this kind is very different from storing other
types of comparable data.  On the one hand, we have the advantage
that, because the characters are integers, we can use them as indices
into arrays. On the other hand, because strings have variable length,
comparison of two strings is not a constant time operations.  In fact,
comparing two strings $s_1$ and $s_2$ takes time
$O(\min(|s_1|,|s_2|\})$, where $|s|$ denotes the length of the string
$s$.

\section{Two Basic Representations}

In most programming languages, strings are built in as part of the language
and they take one of two basic representations, both of which involve storing  the characters of the string in an array.  Both representations are illustrated in \figref{strings}.

In the \emph{null-terminated representation}, strings are represented
as (a pointer to) an array of characters that ends with the special
null terminator $\$$.  This representation is used, for example, in the
C and C++  programming languages.  This representation is fairly straightforward.  Any character of the string can be accessed by its index in constant time.  Computing the length, $|s|$, of a string $s$ takes $O(|s|)$ time since we have to walk through the array, one character at a time until we find the null terminator.

Less common is the \emph{pointer/length representation} in which a
string is represented as (a pointer to) an array of characters along
with a integer that stores the length of the string.  The pointer/length
representation is a little more flexible and more efficient for some
operations.

For example, in the pointer/length representation, determining the length
of the string takes only constant time, since it is already stored.  More
importantly, it is possible to extract any substring in constant time:
If a string $s$ is represented as $(p,\ell)$ where $p$ is the starting
location and $\ell$ is the length, then we can extract the substring
$s_i,s_{i+1},\ldots,s_{i+m}$ by creating the pair $(p+i,m)$.  Several of
the data structures in this chapter will use the pointer/length representation of strings.

\begin{figure}
  \begin{center}
    \includegraphics{figs/strings}
  \end{center}
  \caption{The string ``grain-fed organic'' represented in both the null-terminated and the pointer/length representations.  In the pointer/length representation we can extract the string ``organ'' in constant time.}
  \figlabel{strings}
\end{figure}

\section{Ropes}

A common problem that occurs when developing, for example, a text
editor is how to represent a very long string (text file) so that
operations on the string (insertions, deletions, jumping to a
particular point in the file) can be done efficiently.  In this
section we describe a data structure for just such a problem.
However, we begin by describing a data structure for storing a
sequence of weights.

A \emph{prefix tree} $T$ is a binary tree in which each node $v$
stores two additional values $\wght(v)$ and $\sz(v)$. The value of
$\wght(v)$ is a number that is assigned to a node when it is created
and which may be modified later.  The value of $\sz(v)$ is the sum of
all the weight values stored in the subtree rooted at $v$, i.e.,
\[
   \sz(v) = \sum_{u\in T(v)} \wght(u) \enspace .
\]
It follows immediately that $\sz(v)$ is equal to the sum of the sizes
of $v$'s two children, i.e.,
\begin{equation}
   \sz(v) = \sz(\lft(v)) + \sz(\rght(v)) \enspace . \eqlabel{size}
\end{equation}

When we insert a new node $u$ by making it a child of some node
already in $T$, the only $\sz$ values that change are those on the
path from $u$ to the root of $T$.  Therefore, we can perform such an
insertion in time proportional to the depth of node $u$.  Furthermore,
because of identity \eqref{size}, if all the $\sz$ values in $T$ are
correct, it is not difficult to implement a left or right rotation so
that the $\sz$ values remain correct after the rotation.  Therefore, by
using the treap rebalancing scheme (\secref{treaps}), we can maintain
a prefix-tree under insertions and deletions in $O(\log n)$ expected
time per operation.

Notice that, just as a binary search tree represents its elements in
sorted order, a prefix tree implicitly represents a sequence of
weights that are the weights of nodes we encounter while travering $T$
using an in-order (left-to-right) traversal.  Let $u_1,\ldots,u_n$ be
the nodes of $T$ in left-to-right order.  We can use prefix-trees to
perform searches on the set 
\[
W = \left\{ w_i : w_i = \sum_{j=1}^i \wght(u_i) \right\} \enspace .
\]
That is, we can find the smallest value of $i$ such that $w_i\ge x$
for any query value $x$.  To execute this kind of query we begin our
search at the root of $T$.  When the search reaches some node $u$,
there are three cases

\begin{enumerate}
\item $x < \wght(\lft(u))$.  In this case, we continue the search in
  $\lft(u)$.

\item $\wght(\lft(u)) \le x < \wght(\lft(u)) + \wght(u)$.  In this
  case, $u$ is the node we are searching for, so we report it.

\item $\wght(\lft(u)) + \wght(u) \le x$.  In this case, we search
  for the value $x - \wght(\lft(u)) + \wght(u)$ in $\rght(u)$.
\end{enumerate}

Since each step of this search only takes constant time, the overall
search time is proportional to the length of the path we follow.
Therefore, if $T$ is rebalanced as a treap then the expected search
time is $O(\log M)$.

Furthermore, we can support \textsc{Split} and \textsc{Join}
operations in $O(\log n)$ time using prefix trees.  Given a value $x$,
we can a prefix tree into two trees, where one tree contains all nodes
$u_i$ such that
\[ \sum_{j=1}^i\wght(u_i) \le x
\]
 and the other tree contains all the remaining nodes.  Given two
prefix trees $T_1$ and $T_2$ whose nodes in left-to-right order are
$u_1,\ldots,u_n$ and $v_1,\ldots,v_m$ we can create a new tree $T'$
whose nodes in left-to-right order are
$u_1,\ldots,u_n,v_1,\ldots,v_n$.

Next, consider how we could use a prefix-tree to store a very long
string $s=c_1,\ldots, c_M$ so that it supports the following
operations.

\begin{enumerate}
\item $\textsc{Insert}(i, s')$. Insert the string $s'$ beginning at
position $s_i$ in the string $S$, to form a new string
$c_1,\ldots,c_i\circ s'\circ c_{i+1},\ldots,c_M$.\footnote{Here, and
throughout, $s_1\circ s_2$ denotes the concatenation of strings $s_1$
and $s_2$.}

\item $\textsc{Delete}(i, l)$.  Delete the substring
$c_i,\ldots,c_{i+l-1}$ from $S$ to form a new string
$c_1,\ldots,c_{i-1},c_{i+l},\ldots,c_M$.

\item $\textsc{Report}(i, l)$.  Output the string $c_i,\ldots,c_{i+l-1}$.
\end{enumerate}

To implement these operations, we use a prefix-tree in which each node
$u$ has an extra field, $\strng(u)$.  In this implementation,
$\wght(u)$ is always equal to the length of $\strng(u)$ and we can
reconstruct the string $S$ by concatenating $\strng(u_1)\circ
\strng(u_2)\circ\cdots\circ\strng(u_n)$.  From this it follows that we
can find the character at position $i$ in $S$ by searching for the
value $i$ in the prefix tree, which will give us the node $u$ that
contains the character $c_i$.

To perform an insertion we first create a new node $v$ and set
$\strng(v)=s'$.  We then find the node $u$ that contains $c_i$ and we
split $\strng(u)$ into two parts at $c_i$; one part contains $c_i$ and
the characters that occur before it and the second part contains the
remaining characters (that occur after $c_i$).  We reset $\strng(u)$
so that it contains the first part and create a new node $w$ so that
it $\strng(w)$ contains the second part.  Note that this split can be
done in constant time if each string is represented using the pointer/length representation.

At this point the nodes $v$ and $w$ are not yet attached to $T$.  To
attach $v$, we find the leftmost descendant of $\rght(u)$ and attach
$v$ as the left child of this node.  We then update all nodes on the
path from $v$ to the root of $T$ and perform rotations to rebalance
$T$ according to the treap rebalancing scheme.  Once $v$ is inserted,
we insert $w$ in the same way, i.e., by finding the leftmost
descendant of $\rght(v)$.  The cost of an insertion is clearly
proportional to the length of the search paths for $v$ and $w$, which
are $O(\log M)$ in expectation.

To perform a deletion, we apply the $\textsc{Split}$ operation on treaps
to make three trees.  The tree $T_1$ contains $c_1,\ldots,c_{i-1}$,
the tree $T_2$ contains $c_{i},\ldots,c_{i+l-1}$ and the treap $T_3$
that contains $c_{i+l},\ldots,c_M$.  This may require splitting the two
substrings stored in nodes of $T$ that contain the indices $i$ and $l$,
but the details are straightforward.  We then use the $\textsc{Merge}$
operation of treaps to merge $T_1$ and $T_3$ and discard $T_2$. Since
$\textsc{Split}$ and \textsc{Merge} in treaps each take $O(\log M)$
expected time, the entire delete operation takes in $O(\log M)$
expected time.

To report the string $s_i,\ldots,s_{i+l-1}$ we first search for the
node $u$ that contains $c_i$ and then traverse $T$ starting at node
$u$.  We can then output $s_i,\ldots,s_{i+l-1}$ in $O(l+\log M)$
expected time by doing an in-order traversal of $T$ starting at node
$u$.  The details of this traversal and its analysis are left as an
excercise to the reader.

\begin{thm}
Ropes support the operations $\textsc{Insert}$ and $\textsc{Delete}$
on a string of length $M$ in $O(\log M)$ expected time and
$\textsc{Report}$ in $O(l+\log M)$ expected time.
\end{thm}

\section{Tries}

Next, we consider the problem of storing a collection of strings so
that we can quickly test if a query string is in the collection.  The
most obvious application of such a data structure is in a
spell-checker.

A \emph{trie} is a rooted tree $T$ in which each node has somewhere
between $0$ and $N+1$ children.  All edges of $T$ are assigned labels
in $\{-1,\ldots,N-1\}$ such that all the edges leading the children
of a particular node receive different labels.  Strings are stored as
root-to-leaf paths in the trie so that, if the null-terminated string
$s$ is stored in $T$, then there is a leaf $v$ in $T$ such that the
sequence of edge labels encountered on the path from the root of $T$
to $v$ is precisely the string $s$, including the null terminator.
An example is shown in \figref{trie}.

\begin{figure}
  \begin{center}
    \includegraphics{figs/trie}
  \end{center}
  \caption{A trie containing the strings ``ape'', ``apple'', ``organ'', and ``organism''.}
  \figlabel{trie}
\end{figure}


Notice that it is important that the strings stored in a trie are
null-terminated, so that no string is the prefix of any other string.
If this were not the case, it would be impossible to distinguish, for
example, a trie that contained both ``organism'' and ``organ'' from a
trie that contained just ``organism''.

Implementation-wise, a trie node is represented as an array of
pointers of size $N+1$, which point to the children of the node.  In
this way, the labelling of edges is implicit, since the $i$th element
of the array can represent the edge with label $i-1$.  When we create
a new node, we initialize all of its $N+1$ pointers to nil.

Searching for the string $s$ of length $m$ in a trie $T$ is a simple
operation.  We examine each of the characters of $s$ in turn and
follow the appropriate pointers in the tree.  If at any time we
attempt to follow a pointer that is nil we conclude that $s$ is not
stored in $T$.  Otherwise we reach a leaf $v$ that represents $s$ and
we conclude that $s$ is stored in $T$.  Since the edges at each vertex
are stored in an array and the individual characters of $s$ are
integers, we can follow each pointer in constant time.  Thus, the cost
of searching for $s$ is $O(m)$.

Insertion into a trie is not any more difficult.  We simply the follow
the search path for $s$, and any time we encounter a nil pointer we
create a new node.  Since a trie node has size $O(N)$, this insertion
procedure runs in $O(mN)$ time.

Deletion from a trie is again very similar. We search for the leaf $v$
that represents $s$.  Once we have found it, we delete all nodes on
the path from $s$ to the root of $T$ until we reach a node with more
than 1 child.  This algorithm is easily seen to run in $O(mN)$ time.

If a trie holds a set of strings $S$ which have a total length of $M$
then the total size of the trie is $O(MN)$.  This follows from the
fact that each character of each string results in the creation of at
most 1 trie node.

\begin{thm}
Let $s$ be a string of length $m$.  Tries support insertion or
deletion of $s$ in $O(mN)$ time and searching for $s$ in $O(m)$ time.
If $M$ is the total length of all strings stored in a trie then the
storage used is $O(MN)$.
\end{thm}


\section{Patricia Trees}

A \emph{Patricia tree} (a.k.a.\ a \emph{compressed trie}) is a simple
variant on a trie in which any path whose interior vertices all have only
one child is compressed into a single edge.  For this to make sense,
we now label the edges of the trie with strings, so that the string
corresponding to a leaf $v$ is the concatenation of all the edge labels
we encounter on the path from the root of $T$ to $v$. 
An example is given in \figref{patricia}.

\begin{figure}
  \begin{center}
    \includegraphics{figs/patricia}
  \end{center}
  \caption{A Patricia tree containing the strings ``ape'', ``apple'', ``organ'', and ``organism''.}
  \figlabel{patricia}
\end{figure}

The edge labels of a Patricia tree are represented using the
pointer/length representation for strings.  As with tries, and for the
same reason, it is important that the strings stored in a patricia tree
are null-terminated.  See \figref{patricia-2} for an example.

\begin{figure}
  \begin{center}
    \includegraphics{figs/patricia-2}
  \end{center}
  \caption{The edge labels of a Patricia tree use the pointer/length representation.}
  \figlabel{patricia-2}
\end{figure}

Searching for a string $s$ in a Patricia tree is similar to searching
in a trie, except that when the search traverses an edge it checks the
edge label against a whole substring of $s$, not just a single character.
If the substring matches, the edge is traversed. If there is a mismatch,
the search fails without finding $s$.  If the search uses up all the
characters of $s$, then it succeeds by reaching a leaf corresponding
to $s$.  In any case, this takes $O(|s|)$ time.

Descriptions of the insertion and deletion algorithms follow below.
\figref{patricia-adddel} illustrates the actions of these algorithms.
In this figure, a Patricia tree that initially stores only the string
``orange'' has the string ``organism'' added to it and then has the string
``orange'' deleted from it.

\begin{figure}
  \begin{center}
    \includegraphics{figs/patricia-adddel}
  \end{center}
  \caption{The edge labels of a Patricia tree use the pointer/length representation.}
  \figlabel{patricia-adddel}
\end{figure}

Inserting a string $s$ into a Patricia tree is similar to searching
right up until the point where the search gets stuck because $s$ is not
in the tree.  If the search gets stuck in the middle of an edge, $e$,
then $e$ is split into two new edges joined by a new node, $u$, and the
remainder of the string $s$ becomes the edge label of the edge leading
from $u$ to a newly created leaf.  If the search gets stuck at a node,
$u$, then the remainder of $s$ is used as an edge label for a new edge
leading from $u$ to a newly created leaf.  This takes $O(|s|+N)$ time,
since it involves a search for $s$ followed by the creation of two new
nodes, each of size $O(N)$.

Removing a string $s$ from a Patricia tree is the opposite of insertion.
We first locate the leaf corresponding to $s$ and remove it from the
tree.  If the parent, $u$, of this leaf is left with only one child,
$w$, then we also remove the parent and replace it with a single edge,
$e$, joining $u$'s parent to $w$.  The label for $e$ is obtained in
constant time by extending the edge label that was previously on the
edge $uw$. How long this takes depends on how long it takes to delete
two nodes of size $N$.  If we consider this deletion to be a constant
time operation, then running time is $O(|s|)$.  If we consider this
deletion to take $O(N)$ time, then the running time is $O(|s|+N)$.

An important difference between Patricia trees and tries is that
Patricia trees contain no nodes with only one child.  Every node is
either a leaf or has at least two children.  This immediately implies
that the number of internal (non-leaf) nodes does not exceed the number of
leaves.  Now, recall that each leaf corresponds to a string that is
stored in the Patricia tree so if the Patricia tree stores $n$
strings, the total storage used by nodes is $O(nN)$.  Of course, this
requires that we also store the strings separately at a cost of $O(M)$

\begin{thm}
Let $s$ be a string of length $m$.  Patricia trees support insertion
or deletion of $s$ in $O(m+N)$ time and searching for $s$ in $O(m)$
time.  If $M$ is the total length of all strings and $n$ is the number
of all strings stored in a Patricia tree then the storage used is
$O(nN+M)$.
\end{thm}

In addition to the operations described in the preceding theorem, Patricia trees also support \emph{prefix matches}; they can return a list of all strings that have some not-null-terminated string $s$ as a prefix.  This is done by searching for $s$ in the usual way until running out of characters in $s$.  At this point, every leaf in the subtree that the search ended at corresponds to a string that starts with $s$.  Since every internal node has at least two children, this subtree can be traversed in $O(k)$ time, where $k$ is the number of leaves in the subtree.

If we are only interested in reporting one string that has $s$ as a
prefix, we can look at the edge label of the last edge on the search
path for $s$.  This edge label is represented using the pointer/length
representation and the pointer points to a longer string that has
$s$ as a prefix (consider, for example, the edge labelled ``ple\$''
in \figref{patricia-2}, whose pointer points to the second `p' of
``apple\$'').  This edge label can therefore be extended backwards to
report the actual string that contains this label.

\begin{thm}
For a query string $s$, a Patricia tree can report one string that has $s$ as a prefix in $O(|s|)$ time and can report all strings that have $s$ as a prefix in $O(|s|+k)$ time, where $k$ is the number of strings that have $s$ as a prefix.
\end{thm}

\section{Suffix Trees}

Suppose we have a large body of text and we would like a data structure
that allows us to query if particular strings occurs in the text.  Given a
string $t$ of length $n$, we can insert each of the $n+1$ suffixes of
$t$ into a Patricia tree.  We call the resulting tree the \emph{suffix
tree} for $t$.  Now, if we want to know if some string $s$ occurs in $t$
we need only do a prefix search for $s$ in the Patricia tree.  Thus,
we can test if $s$ occurs in $t$ in $O(|s|)$ time.

What is the storage required by the suffix tree for $T$?  Since we
only insert $n$ suffixes, the storage required by tree nodes is
$O(nN)$.  Furthermore, recall that the labels on edges of $T$ are
represented as pointers into the strings that were inserted into $T$.
However, every string that is inserted into $T$ is a suffix of $t$, so
all labels can be represented as pointers into a single copy of $t$, so the total spaced used to store $t$ and all its edge labels is only $O(n)$.
Thus, the total storage used by a suffix tree is $O(nN)$.

The cost of constructing the suffix tree for $t$ can be split into
two parts:  The cost of creating and initializing new nodes, which
is clearly $O(nN)$ because there are only $O(n)$ nodes; and the
cost of following paths, which is clearly $O(n^2)$.

\begin{thm}\thmlabel{suffix-tree-slow}
The suffix tree for a string $t$ of length $n$ can be constructed in
$O(nN+n^2)$ time and uses $O(nN)$ storage.  The suffix tree can be
used to determine if any string $s$ of length $m$ is a substring of $t$
in $O(m)$ time.  The suffix tree can also report the locations of all
occurrences of $s$ in $t$ in $O(m+k)$ time, where $k$ is the number of
occurrences of $s$ in $t$.
\end{thm}

The construction time in \thmref{suffix-tree-slow} is non-optimal.
References to $O(n)$ time suffix tree construction algorithms are given
in \secref{strings-discussion}.


\section{Suffix Arrays*}

The \emph{suffix array} $A_1,\ldots,A_n$ of a string $t=t_1,\ldots,t_n$ lists
the suffixes of $t$ in lexicographically increasing order.  That is,
$A_i$ is the index such that $t_{A_{i}},\ldots,t_m$ has rank $i$ among
all suffixes of $A$.  

For example, consider the string ``counterrevoluationary\$'':
\begin{center}
  \begin{tabular}{|c|c|c|c|c|c|c|c|c|c|c|c|c|c|c|c|c|c|c|c|c|}\hline
     1&2&3&4&5&6&7&8&9&10&11&12&13&14&15&16&17&18&19&20&21\\\hline
     c&o&u&n&t&e&r&r&e&v&o&l&u&t&i&o&n&a&r&y&\$\\\hline
  \end{tabular}
\end{center}
The suffix array for $t$ is
$A=\langle18,1,6,9,15,12,17,4,11,16,2,8,7,19,5,14,3,13,10,20,21\rangle$.  This is hard to verify, so here is a table that helps:

%\begin{center}
%  \begin{tabular}{|c|c|c|c|c|c|c|c|c|c|c|c|c|c|c|c|c|c|c|c|c|}\hline
%18& 1& 6& 9& 15& 12& 17& 4& 11& 16& 2& 8& 7& 19& 5& 14& 3& 13& 10& 20& 21\\\hline
%    \begin{turn}{-90}ary\$\end{turn}&
%    \begin{turn}{-90}counterrevolutionary\$\end{turn}&
%    \begin{turn}{-90}errevolutionary\$\end{turn}&
%    \begin{turn}{-90}evolutionary\$\end{turn}&
%    \begin{turn}{-90}ionary\$\end{turn}&
%    \begin{turn}{-90}lutionary\$\end{turn}&
%    \begin{turn}{-90}nary\$\end{turn}&
%    \begin{turn}{-90}nterrevolutionary\$\end{turn}&
%    \begin{turn}{-90}olutionary\$\end{turn}&
%    \begin{turn}{-90}onary\$\end{turn}&
%    \begin{turn}{-90}ounterrevolutionary\$\end{turn}&
%    \begin{turn}{-90}revolutionary\$\end{turn}&
%    \begin{turn}{-90}rrevolutionary\$\end{turn}&
%    \begin{turn}{-90}ry\$\end{turn}&
%    \begin{turn}{-90}terrevolutionary\$\end{turn}&
%    \begin{turn}{-90}tionary\$\end{turn}&
%    \begin{turn}{-90}unterrevolutionary\$\end{turn}&
%    \begin{turn}{-90}utionary\$\end{turn}&
%    \begin{turn}{-90}volutionary\$\end{turn}&
%    \begin{turn}{-90}y\$\end{turn}&
%    \begin{turn}{-90}\$\end{turn}\\\hline
%  \end{tabular}
%\end{center}

\begin{center}
  \begin{tabular}{|r|r|l|}\hline
    $i$ & $A_i$ & $t_i,\ldots,t_n$ \\\hline
1&18&ary\$\\
2&1&counterrevolutionary\$\\
3&6&errevolutionary\$\\
4&9&evolutionary\$\\
5&15&ionary\$\\
6&12&lutionary\$\\
7&17&nary\$\\
8&4&nterrevolutionary\$\\
9&11&olutionary\$\\
10&16&onary\$\\
11&2&ounterrevolutionary\$\\
12&8&revolutionary\$\\
13&7&rrevolutionary\$\\
14&19&ry\$\\
15&5&terrevolutionary\$\\
16&14&tionary\$\\
17&3&unterrevolutionary\$\\
18&13&utionary\$\\
19&10&volutionary\$\\
20&20&y\$\\
21&21&\$\\\hline
\end{tabular}
\end{center}

Given a suffix-array $A=A_1,\ldots,A_n$ for $t$ and a not-null-terminated
query string $s$ one can do binary search in $O(|s|\log n)$ time to
determine whether $s$ occurs in $t$;  binary search uses $O(\log n)$
comparison and each comparison takes $O(|s|)$ time.

We can do searches a little faster---in $O(|s|+\log n)$ time---if,
in addition to the suffix array, we have a little more information.
Let $s_i=t_{A_i}, t_{A_i+1},\ldots,t_{n}$, i.e., $s_i$ is the string in
the $i$th row of the preceding table.

Suppose, that we have access to a table $\ell$, where, for any
pair of indices $(i,j)$ with $1\le i\le j\le n$, $\ell_{i,j}$ is
the length of the longest common prefix of $s_i$ and $s_j$. In our
running example, $\ell_{9,11}=1$, since $s_9=\text{``olutionary\$}''$
and $s_{11}=\text{``ounterrevolutionary''}$ have the first character
``o'' in common but differ on their second character.  Notice that
this also implies that $s_{10}$ starts with the same letter as $s_9$
and $s_{11}$, otherwise $s_{10}$ would not be between $s_9$ and $s_{11}$
in sorted order.  More generally, if $\ell_{i,j}=r$, then the suffixes
$s_i,s_{i+1},\ldots,s_{j}$ all start with the same $r$ characters.

With the extra \emph{longest common prefix} information provided by $\ell$, binary search on the suffix array can be sped up.  Suppose that the binary search has already run for a while and has concluded that a matching string---if it is exists---is among $s_i,s_{i+1},\ldots,s_j$ and we have verified that
the first $r_i$ characters of $s$ match the first $r_i$ characters of $s_i$ and the first $r_j$ characters of $s$ match the first $r_j$ characters of $s_j$.

Suppose, without loss of generality, that $r_i \ge r_j$; that is, $s$ matches at least as many characters in $s_i$ as in $s_j$.  Now, consider 


   Now we consider the string $s_m$, where $m=\lceil(i+1)/2\rceil$.  We start matching the characters at
indices $r+1$, $r+2$, $r+3$, and so


We know that the first $\ell_{i,j}$ characters of every string in this set are the same, and they all match 


\subsection{Constructing a suffix array in $O(n)$ time}


TODO: Describe the LCP structure, the $O(|s|+\log M)$ time search algorithm,
and K\"arkk\"ainen and Sanders $O(M+N)$ time construction algorithm.

\begin{thm}
The suffix array for a string $t$ of length $M$ can be constructed in
$O(M+N)$ time and uses $O(M)$ storage.  The suffix array can be used to
determine if any string $s$ is a substring of $t$ in $O(|s|+\log M)$ time.
\end{thm}


\section{Ternary Tries}

The last three sections discussed data structures for storing strings
where the running times of the operations were independent of the
number of strings actually stored in the data structure.  This is not
to be underestimated.  Consider that, if we store the collected works
of Shakespeare in a suffix tree, it is possible to test if the word
``warble'' appears anywhere in the text by following 7 pointers.

This result is possible because the individual characters of strings
are used as indices into arrays.  Unfortunately, this approach has
two drawbacks: The first is that it requires that the characters be
integers.  The second is that $N$, the size of the alphabet becomes a
factor in the storage and running time.

To avoid having $N$ play a role in the running time, we can restrict
ourselves to algorithms that only perform comparisons between
characters.  One way to do this would be to simply store our strings
in a binary search tree, in lexicographic order.  Then a search for
the string $s$ could be done with $O(\log n)$ string comparison, each
of which takes $O(|s|)$ time, for a total of $O(|s|\log n)$.

Another way to reduce the dependence on $N$ is to store child pointers
in a binary search tree.  A \emph{ternary} trie is a trie in which
pointers to the children of each of node $v$ are stored in a binary
search tree.  If we use a balanced binary search trees for this, then it
is clear that the insertion, deletion and search times for the string
$s$ are $O(|s|\log N)$.  If $N$ is large, we can do significantly
better than this, by using a different method of balancing our search
trees.

Note that a ternary trie can be viewed as a single tree in which each
node has up to 3 children (hence the name).  Each node $v$ has a left
child, $\lft(v)$, a right child, $\rght(v)$, a middle child, $\md(v)$,
and is labelled with a character, denoted $\m(v)$.  A root-to-leaf path
$P$ in the tree corresponds to exactly one string, which is obtained by
concatenating the characters $\m(v)$ for each node $v$ whose successor
in $P$ is a middle child.

Now, suppose that every node $v$ of our ternary trie has the balance
property $|L(\rght(v))|\le |L(v)|/2$ and $|L(\lft(v))|\le |L(v)|/2$,
where $L(v)$ denotes the set of leaves in the subtree rooted at $v$.
Then the length of any root-to-leaf path $P$ is at most $O(|s|+\log
n)$ where $|s|$ is the length of the string represented by $P$.  This
is easy to see, because every time the path traverses an edge
represented by a $\md$ pointer the number of symbols in $s$ that are
unmatched decreases by one and every time the path traverses an edge
represented by a $\lft$ or $\rght$ pointer the number of leaves in the
current subtree decreases a factor of at least $1/2$.

Given a set $S$ of strings, constructing a ternary trie with the above
balance property is easily done.  We first sort all our input strings
and then look at the first character, $m$, of the string whose rank is
$n/2$ in the sorted order.  We then create a new node $u$ with label
$m$ to be the root of the ternary trie.  We collect all the strings
that begin with $m$, strip off their first character, and recursively
insert these into the middle child of $u$.  Finally, we recursively
insert all strings beginning with characters less than $u$ in the left
child of $u$ and all strings beginning with characters greater than
$u$ in the right child of $u$.

Ignoring the cost of sorting and recursive calls, the above algorithm
can easily be implemented to run in $O(n')$ time, where $n'$ is the
number of strings beginning with $m$.  However, during this operation,
we strip off $n'$ characters from strings and never have to look at
these again.  Therefore the total running time, including recursive
calls, is not more than $O(M)$.

\begin{thm}
After sorting the input strings, a ternary trie can be constructed in
$O(M)$ time and can search for a string $s$ in $O(|s|+\log n)$ time.
\end{thm}

TODO: Show how to make suffix trees dynamic, using randomization.

\section{Quad-Trees}

An interesting generalization of tries occurs when we want to store
two (or more) dimensional data.  Imagine that we want to store real
values in the unit square $[0,1)^2$, where each point $(x,y)$ is
represented by two binary strings $x_1,x_2,x_3,\ldots$ and
$y_1,y_2,y_3,\ldots$ where $x_i$ (respectively $y_i$) is the
$i$th bit in the binary expansion of $x$ (respectively, $y$).  When we
consider the most-significant bits of the binary expansion of $x$ and
$y$, four cases can occur:


\[(x,y) = \begin{array}{|c|c|}\hline
(.0\ldots,.1\ldots) & (.1\ldots,.1\ldots) \\ \hline
(.0\ldots,.0\ldots) & (.1\ldots,.0\ldots) \\ \hline
\end{array}
\]

Thus, it is natural that we store our points in a tree where each node
has up to four children.  In fact, if we treat $(x,y)$ as the string
$(x_1,y_1)(x_2,y_2)(x_3,y_3),\ldots,$ over the alphabet
$\Sigma=\{(0,0),(0,1),(1,0),(1,1)\}$ and store $(x,y)$ in a trie then
this is exactly what happens.  The tree that we obtain is called a
\emph{quad tree}.

Quad trees have many applications because they preserve spatial
relationships very well, much in the way that tries preserve prefix
relationships between strings.  As a simple example, we can consider
queries of the form: report all points in the rectangle with bottom
left corner $[.0101010,.1111110]$ and top right corner
$[.0101011,.1111111]$.  To answer such a query, we simply follow the
path for $(0,1)(1,1)(0,1)(1,1)(0,1)(1,1)$ in the quadtree (trie) and
report all leaves of the subtree we find.

Quadtrees can be generalized in many ways.  Instead of considering
binary expansions of the $x$ and $y$ coordinates we can consider
$m$-ary expansions, in which case we obtain a tree in which each node
has up to $m^2$ children.  If, instead of points in the unit square,
we have points in the unit hypercube in $\mathbb{R}^d$ then we obtain
a tree in which each node has $2^d$ children.  If we use a Patricia
tree in place of a trie then we obtain a \emph{compressed quadtree}
which, like the Patricia tree uses only $O(n+M)$ storage where $M$ is
the total size of (the binary expansion of) all points stored in the
quadtree.  

\section{Discussion and References}
\seclabel{strings-discussion}

Prefix-trees seem be part of the computer science folklore, though
they are not always implemented using treaps. The first documented use
of a prefix-tree is unclear.

Ropes (sometimes called cords) are described by Boehm \etal\ \cite{bap95}
as an alternative to strings represented as arrays.  They are so useful
that they have made it into the SGI implementation of the C++ Standard
Template Library \cite{x}.

Tries, also known as digital search trees, have a very long history.
Knuth \cite{k73c} is a good resource to help unravel some of their
history.  Patricia trees were described and given their name by
Morrison \cite{m68}.  PATRICIA is an acronym for Practical Algorithm
To Retrieve Information Coded In Alphanumeric.

Suffix trees were introduced by Weiner \cite{w73}, who also showed
that, if $N$, the size of the alphabet, is constant then there is an
$O(M)$ time algorithm for their construction.  Since then, several
other simplified $O(M)$ time construction algorithms for suffix trees
have been presented \cite{cs85,m76}.  Recently, Farach \cite{f97} gave
an $O(M)$ time algorithm for the case where $N$ is as large as $M$,
the length of the string.

Ternary tries also have a long history, dating back at least until
1964, when Clampett \cite{c64} suggested storing the children of trie
nodes in a binary search tree.  Mehlhorn \cite{m79} shows that ternary
tries can be rebalanced so that insertions and deletions can be done
in $O(|s|+\log n)$ time.  Sleator and Tarjan \cite{st85} showed that,
if the splay heuristic (\secref{splay-trees}) is applied to ternary
tries, then the cost of a sequence of $n$ operations involving a set
of strings whose total length is $M$ is $O(M+n\log n)$.  Furthermore,
with splaying, a sequence of $n$ searches can be executed in $O(M+
nH)$ time, where $H$ is the empirical entropy (\secref{entropy}) of
the access sequence.  Vaishnavi \cite{v84} and Bentley and Saxe
\cite{bs79} arrived at ternary tries through the geometric problem of
storing a set of vectors in $d$-dimensions.  Finally, ternary tries
were recently revisited by Bentley and Sedgewick \cite{bs97}.

Samet's books \cite{s90,sXX,sYY} provide an encyclopedic treatment of
quadtrees and their applications.

\bibliography{tds}
