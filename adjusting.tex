\chapter{Self-Adjusting Data Structures}\chaplabel{adjusting}
\bibliographystyle{plain}

\chapref{entropy} describes a data structure that is able to achieve
an expected query time that is proportional to the entropy of the
query distribution.  The most interesting aspect of this data
structure is that it does need to know the query distribution in
advance.  Rather, the data structure adjusts itself using information
gathered from previous queries.

In this chapter we study several other data structures that adjust
themselves in the same way.  All of these data structures are
efficient only in an amortized sense.  That is, individual operations
take may take a large amount of time, but over a sequence of many
operations, the amortized (average) time per operation is still small.

The advantage of self-organizing data structures is that they can
often perform ``natural'' sequences of accesses much faster than their
worst-case counterparts.  Examples of such natural sequences including
accessing each of the elements in order, using the data structure like
a stack or queue and performing many consecutive accesses to the same
element.

In other sections of this book we have analyzed the amortized cost of
operations on data structures using the accounting method, in which we
manipulate credits to bound the amortized cost of operations.  A
different method of amortized analysis is the \emph{potential method}.
For any data structure $D$, we define a potential $\Phi(D)$.  If we
let $D_i$ denote our data structure after the $i$th operation and let
$C_i$ denote the cost (running time) of the $i$th operation then the
\emph{amortized cost} of the $i$th operation is defined as
\[
  A_i = C_i+\Phi(D_i)=\Phi(D_{i-1}) \enspace .
\] 
Note that this is simply a definition and, since $\Phi$ is arbitrary,
$A_i$ has nothing to do with the real cost $C_i$.  However, the
the cost of a sequence of $m$ operations is 
\begin{eqnarray*}
  \sum_{i=1}^m C_i 
	& = & \sum_{i=1}^m \left(C_i + \Phi(D_i)-\Phi(D_{i-1})\right) -
		\sum_{i=1}^m \left(\Phi(D_i)-\Phi(D_{i-1})\right) \\
	& = & \sum_{i=1}^m A_i -
		\sum_{i=1}^m \left(\Phi(D_i)-\Phi(D_{i-1})\right) \\
	& = & \sum_{i=1}^m A_i + \Phi(D_0) - \Phi(D_m) \enspace .
\end{eqnarray*}

The quantity $\Phi(D_0) - \Phi(D_m)$ is referred to as the \emph{net
drop in potential}.  Thus, if we can provide a bound $D$ on $A_i$,
then the total cost of $m$ operations is $mD$ plus the net drop in
potential.  

The potential method is equivalent to the accounting method.  Anything
that can be proven with the potential method can be proven with the
accounting method. (Just think of $\Phi(D)$ as the number of credits
the data structure stores, with a negative value indicating that the
data structure owes credits) However, once a potential function is
defined, mathematical manipulations that can lead to very compact
proofs are convenient.  Of course, the rub is in finding the right
potential function.

\section{Splay Trees}\seclabel{splay-trees}

One of the first-ever self-modifying data structures is the
\emph{splay tree}.  We begin our discussion of splay trees with the
definition and analysis of the splay operation. A splay tree is a
binary search tree that is augmented with the splay operation. Given a
binary search tree $T$ rooted at $r$ and a node $x$ in $T$, splaying
node $x$ is a sequence of left and right rotations performed on the
tree until $x$ is the root of $T$. This sequence of rotations is
governed by the following rules: (Note: a node is a left
(respectively, right) child if it is the {left (respectively, right)
child of its parent) In what follows, $y$ is the parent of $x$ and $z$
is the parent of $y$ (should it exist).

\begin{figure}
\begin{center}\begin{tabular}{ccc}
\includegraphics{figs/zig-a} & \raisebox{1cm}{zig} & \includegraphics{figs/zig-b} \\[1cm]
\includegraphics{figs/zigzig-a} & \raisebox{1cm}{zig-zig} & \includegraphics{figs/zigzig-b} \\[1cm]
\includegraphics{figs/zigzag-a} & \raisebox{1cm}{zig-zag} & \includegraphics{figs/zigzag-b}
\end{tabular}\end{center}
\caption{Splay tree.}
\end{figure}

\begin{description}
\item[1. Zig step:] If $y$ is the root of the tree, then rotate $x$ so
that it becomes the root.

\item[2. Zig Zig step:] The node $y$ is not the root. Both $x$ and $y$
are left (or right) children, then first rotate $y$, so that $z$ and
$x$ are children of $y$, then rotate $x$ so that it becomes the parent
of $y$.

\item[3. Zig Zag step:] Node $y$ is not the root. If $x$ is a right
child and $y$ is a left child or vice-versa, rotate $x$ twice. After
one rotation it is the parent of $y$ and after the second rotation,
both $y$ and $z$ are children of $x$.
\end{description}

We are now ready to define the potential of a splay tree. Let $T$ be a
binary search tree on $n$ nodes and let the nodes be numbered from 1
to $n$. Assume that each node $i$ has a fixed positive weight
$w(i)$. Define the size $s(x)$ of a node $x$ to be the sum of the
weights of all the nodes in the subtree rooted at $x$, including the
weight of $x$. The rank $r(x)$ of a node $x$ is $\log(s(x))$. By
definition, the rank of a node is at least as large as the rank of any
of its decendants. The potential of a given tree $T$ is $\Phi(T) =
\sum_{i=1}^n \log(s(i))$.

All of the properties of a splay tree are derived from the following
lemma:

\begin{lem}\lemlabel{splay}
The amortized time to splay a binary search tree $T$ with root $t$ at
a node $x$ is at most $3(r(t)-r(x)) + 1 = O(\log(s(t))/\log(s(x)))$
\end{lem}

\begin{proof}
If node $x$ is the root, then the lemma holds trivially. If node $x$ is not the root, then a number of rotations are performed according to the rules of splaying. We first show that the lemma holds after each step (zig, zigzig or zigzag) is performed. Let $r_0(i)$, $r_1(i)$ and $s_0(i)$, $s_1(i)$ represent the rank and size of node $i$ before and after the first step, respectively. Recall that the amortized cost of an operation $a = t + \Phi_1(T) - \Phi_0(T)$ where $t$ is the actual cost of the operation, and $\Phi_0$ and $\Phi_1$ are the potentials before and after the step.

\begin{description}
\item[1. Zig step:] In this case, $y$ is the root of the tree and a
single rotation is performed to make $x$ the root of the tree. In the
whole splay operation, this step is only ever executed once. The
actual cost of the rotation is 1. We show that the amortized cost is
at most $3(r_1(x) - r_0(x)) + 1$.

\begin{eqnarray}
\mbox{amortized cost} & =  & 1 + r_1(x) + r_1(y) - r_0(x) -r_0(y) 
	\eqlabel{a1}\\
                      &\leq& 1 + r_1(x) - r_0(x) \eqlabel{a2} \\
                      &\leq& 1 + 3(r_1(x) - r_0(x)) \eqlabel{a3}
\end{eqnarray}

\Eqref{a1} follows from the definition of amortized cost and
the fact that the only ranks that changed are the ranks at $x$ and
$y$. The ranks of all other nodes in the tree remain the same since
the nodes in their subtrees remain the same. Therefore, the change in
potential is determined by the change in the ranks of $x$ and
$y$. \Eqref{a2} holds because prior to the rotation, node $y$
was the root of the tree which implies that $r_0(y) \geq
r_1(y)$. \Eqref{a3} follows since after the rotation, node $x$
is the root the tree, which means that $r_1(x) - r_0(x)$ is positive.

\item[2. Zig Zig step:] The node $y$ is not the root. Both $y$ and $z$
are left (or right) children. In this case, two rotations are
made. The first rotation is at $y$, so that $z$ and $x$ are children
of $y$. The second rotation is at $x$ so that it becomes the parent of
$y$ and grandparent of $z$. Since two rotations are performed, the
actual cost of this step is 2.

\begin{eqnarray}
\mbox{amortized cost} & =  & 2 + r_1(x) + r_1(y) + r_1(z) - r_0(x) -r_0(y) - r_0(z) \eqlabel{b1}\\
                      & =  & 2 + r_1(y) + r_1(z) - r_0(x) - r_0(y) 
	\eqlabel{b2}\\
                      &\leq& 2 + r_1(x) + r_1(z) - 2r_0(x)) \eqlabel{b3}\\
\end{eqnarray}

After the two rotations, the only ranks that change are the ranks of
$x$, $y$ and $z$. Therefore, the change in potential is defined by the
change in their ranks. \Eqref{b1} follows from the definition of
amortized cost. Now, prior to the two rotations, $x$ is child of $y$
which is a child of $z$ and after the two rotations, $z$ is a child of
$y$ which is a child of $x$. This implies that $r_1(x) =
r_0(z)$. \Eqref{b2} follows. Similarly, \eqref{b3} follows
since $r_1(x) \geq r_1(y)$ and $r_0(x) \leq r_0(y)$. What remains to
be shown is that $2 + r_1(x) + r_1(z) - 2r_0(x)) \leq 3(r_1(x) -
r_0(x))$.

With a little manipulation, we see that this is equivalent to showing
that $r_1(z) + r_0(x) - 2r_1(x) \leq -2$.

\begin{eqnarray}
r_1(z) + r_0(x) - 2r_1(x) & = & \log(s_1(z)/s_1(x)) + \log(s_0(x)/s_1(x))
	\eqlabel{b4}\\
                       & = & \log((s_1(z)/s_1(x))*(s_0(x)/s_1(x))) 
	\eqlabel{b5}\\ 
                       &\leq& \log(1/4)
	\eqlabel{b6}\\
                       & = & -2
\end{eqnarray}

\Eqref{b4} simply follows from the definition of ranks. Notice that
$s_1(x) \geq s_1(z) + s_0(x)$. This implies that $s_1(z)/s_1(x) +
s_0(x)/s_1(x) \leq 1$. If we let $b= s_1(z)/s_1(x)$ and
$c=s_0(x)/s_1(x)$, we note that $b+c\leq 1$ and the expression in
\eqref{b5} becomes $\log(bc)$. Elementary calculus shows that given
$b+c \leq 1$, the product $bc$ is maximized when $b=c=1/2$. This gives
\eqref{b6}.

 
\item[3. Zig Zag step:] Node $y$ is not the root. If $x$ is a right
child and $y$ is a left child or vice-versa, rotate $x$ twice. After
one rotation it is the parent of $y$ and after the second rotation,
both $y$ and $z$ are children of $x$. Again since two rotations are
performed, the actual cost of the operation is 2. Let us look at what
the amortized cost is.

\begin{eqnarray}
\mbox{amortized cost} & =  & 2 + r_1(x) + r_1(y) + r_1(z) - r_0(x) -r_0(y) - r_0(z) \eqlabel{b7}\\
                      & =  & 2 + r_1(y) + r_1(z) - r_0(x) - r_0(y) 
	\eqlabel{b8}\\
                      &\leq& 2 + r_1(y) + r_1(z) - 2r_0(x)) 
	\eqlabel{b9}\\
\end{eqnarray}

After the two rotations, the only ranks that change are the ranks of
$x$, $y$ and $z$. Therefore, the change in potential is again defined
by the change in their ranks. \Eqref{b7} follows from the
definition of amortized cost. Now, prior to the two rotations, $z$ is
the root of the subtree under consideration and after the two
rotations, $x$ is the root. Therefore, we have $r_1(x) =
r_0(z)$. \Eqref{b8} follows. Similarly, $r_0(x) \leq r_0(y)$
since $x$ is a child of $y$ prior to the rotations. What remains to be
shown is that $2 + r_1(y) + r_1(z) - 2r_0(x)) \leq 3(r_1(x) -
r_0(x))$. This is equivalent showing that $r_1(y) + r_1(z) + r_0(x) -
3r_1(x) \leq -2$.

\begin{eqnarray}
r_1(y) + r_1(z) + r_0(x) - 3r_1(x) & = & \log((s_1(y)/s_1(x))*(s_1(z)/s_1(x))) + r_0(x) - r_1(x) \eqlabel{b10}\\
                       &\leq& \log(1/4)
	\eqlabel{b11}\\
                       & = & -2
\end{eqnarray}

The argument is similar to the previous case. \Eqref{b10}
simply follows from the definition of ranks and some manipulation of
logs. The inequality \eqref{b11} follows because $s_1(x)\geq s_1(y) +
s_1(z)$ and $r_0(x) \leq r_1(x)$.
\end{description}

Now, suppose that a splay operation requires $k>1$ steps. Only the kth
step will be a zig step and all the others will either be zigzig or
zigzag steps. From the three cases shown above, the amortized cost of
the whole splay operation is
\[
3r_k(x) - 3r_{k-1}(x) + 1 + \sum_{j=1}^{k-1} (3r_{j}(x) - 3r_{j-1}(x)) =
3r_k(x) - 3r_0(x) + 1 \enspace .
\]  
\end{proof}

If each node is given a weight of 1, the above lemma suggests that the
amortized cost of performing a splay operation is $O(\log n)$ where
$n$ is the size of the tree. However, \lemref{splay} is
surprisingly much more powerful. Notice that the weight of a node was
never really specified. We only stated that it should be a positive
constant. The power comes from the fact that the lemma holds for {\bf
any} assignment of positive weights to the nodes. By carefully
choosing the weights, we can prove many interesting properties of
splay trees. Let us look at some examples.

\begin{thm}[Balance Theorem]
Given a sequence of $m$ accesses into an $n$ node splay tree $T$, the
total access time is $O((m+n)\log n)$.
\end{thm}

\begin{proof}
Assign a weight of $1$ to each node. The maximum value that $s(x)$
can have for any node $x \in T$ is $n$ since the sum of the weights is
$n$. The minimum value of $s(x)$ is $1$. Therefore, the amortized access cost
for any element is at most $3(\log(n) - \log(1)) + 1 = 3\log(n)
+1$. The maximum potential $\Phi_{max}(T)$ that the tree can have is
\[
   \sum_{i=1}^n \log(s(i)) \leq \sum_{i=1}^n \log(n) = n\log n 
\]
The minimum
potential $\Phi_{min}(T)$ that the tree can have is 
\[
   \sum_{i=1}^n \log(s(i)) \geq \sum_{i=1}^n \log(1) = 0 \enspace .
\]
Therefore, the maximum drop in potential over the whole sequence is
$\Phi_{max}(T) - \Phi_{min}(T) = n\log n$.  Thus, the total cost of
accessing the sequence is $O(m\log n + n\log n)$, as required.
\end{proof}

This means that starting from \emph{any} initial binary search tree, if
we access $m > n$ keys, the cost per access will be $O(\log n)$. This
is quite interesting since we can start with a very unbalanced tree,
such as an $n$-node path, and still achieve good behaviour.

%\begin{thm}[Empirical Static Optimality Theorem] 
%Given a sequence of $m$ accesses into an $n$ node splay tree $T$, where
%every node is accessed at least once, then the total access time is
%$O(m + \sum_{i=1}^n q(i)\log(m/q(i))$, where $q(i)$ is the access
%frequency of node $i$.
%\end{thm}
%
%\begin{proof}
%Assign a weight of $q(i)/m$ to each node. The maximum value that
%$s(x)$ can have for any node $x \in T$ is 1 since the sum of the
%weights is 1. The minimum value is $q(i)/m$. Therefore, the amortized
%access cost for any element is at most $3(\log(1) - \log(q(i)/m)) + 1
%= 3\log(m/q(i)) +1$. The maximum potential $\Phi_{max}(T)$ that the
%tree can have is $\sum_{i=1}^n \log(s(i)) \leq \sum_{i=1}^n \log(1) =
%0$. The minimum potential $\Phi_{min}(T)$ that the tree can have is
%$\sum_{i=1}^n \log(s(i)) \geq \sum_{i=1}^n \log(q(i)/m)$. Therefore,
%the maximum drop in potential over the whole sequence is
%$\Phi_{max}(T) - \Phi_{min}(T) = \sum_{i=1}^n \log(q(i)/m)$.
%\end{proof}

\begin{thm}[Static Optimality Theorem] 
Fix some static binary search tree $T^*$ that contains the elements
$1,\ldots,n$, fix an access sequence $a_1,\ldots,a_m$, and let $T^*(n)$
denote the cost of searching for $a_1,\ldots,a_m$ using $T^*$.  Then the
cost of accessing $a_1,\ldots,a_m$ using a splay tree is $O(n\log n + T^*(n))$.
\end{thm}

\begin{proof}
If $d(x)$ is the depth of node $x$ in $T^*$, then we assign a weight of
$w(x)=1/3^{d(x)}$ to the node containing $x$ in the splay tree.  Now, note that the size of the root (and hence any node) is at most 
\[
    \sum_{i=0}^\infty 2^i/3^i = 3 \enspace ,
\]
since $T^*$ has at most $2^i$ nodes with depth $j$ and these each have
weight $1/3^j$.

Now, according to the Access Lemma, the amortized cost of accessing node $x$ is
at most
\[
    3(\log 3 - \log(s(x)) + 1 = O(1) + 3\log(3^{d(x)}) 
    = O(1) + (3\log_2 3)d(x) = O(1+d(x))
\]
\end{proof}

\section{Pairing Heaps}

John Iacono apparently has a mapping between splay trees and pairing
heaps that makes all the analyses carry over.

\section{Queaps}

Next, we consider an implementation of priority queues.  These are
data structures that support the insertion of real-valued priorities.
At any time, the data structure can return a pointer to a node storing
the minimum priority, and given a pointer to a node, that node (and
it's corresponding priority) can be deleted.

For the data structure we describe, the cost to insert an element is
$O(1)$ and the cost to extract the element $x$ is $O(\log(q(x))$,
where $q(x)$ denotes the number of elements that have been in the data
structure longer than $x$.  Thus, if the priority queue is used as a
normal queue (the first element inserted is the first element deleted)
then all operations take constant time.  We say that a data structure
that has this property is \emph{queueish}.

The data structure consists of a balanced tree $T$ and a list $L$.
Refer to \figref{queap}. The tree $T$ is 2-4 tree, that is, all leaves
of $T$ are on the same level and each internal node of $T$ has between
2 and 4 children.  From this definition, it follows immediately that
the $i$th level of $T$ contains at least $2^i$ nodes, so the height of
$T$ is logarithmic in the number of leaves.  Another property of 2-4
trees that is not immediately obvious, but not difficult to prove, is
that it is possible to insert and delete a leaf in $O(1)$ amortized
time.  These insertions and deletions are done by splitting and
merging nodes (see
\figref{twofour}), but any sequence of $n$ insertions and deletions
results only in $O(n)$ of these split and merge operations.

\begin{figure}
\begin{center}\begin{tabular}{ccc}
\includegraphics{figs/twofour-a} & \raisebox{.6cm}{$\Leftrightarrow$} & \includegraphics{figs/twofour-b}
\end{tabular}\end{center}
\caption{Splitting (right to left) and merging (left to right) nodes in a 2-4 tree.}\figlabel{twofour}
\end{figure}

\begin{figure}
\centergraphics{queap}
\caption{A queap.}\figlabel{queap}
\end{figure}

It is important to note that $T$ is not being used as a search tree.
Rather, the leaves of $T$ will store priorities so that the left to
right order of the leaves is the same as the order in which the
corresponding priorities were inserted.

The nodes of $T$ also contain some additional information.  The
\emph{left wall} of $T$ is the path from the root to the leftmost leaf.
Each internal node $v$ of $T$ that is not part the left wall stores a
pointer $\min(v)$ to the leaf in the subtree rooted at $v$ with
minimum priority.  Each internal node $v$ that is part of the left
wall stores a pointer $\min(v)$ to the leaf \emph{not} in the subtree
rooted at $v$ with minimum priority.  Finally, we also store a pointer
to the leaf with minimum priority in $T$ in the variable $\min(T)$.
It is clear that updating the min values at a node that is split or
two nodes that are merged can be done in constant time, so in all
further discussions it is assumed that when nodes are split or merged
the $\min$ values at those nodes are updated appropriately.

The list $L$ is a list of priorities that were inserted after all the
priorities in $T$.  The list is ordered by the order in which the
priorities were inserted, so that the least recently inserted priority
is the first element of $L$.  Along with $L$, we also maintain a
pointer $\min(L)$ to the list node with minimum priority.

To perform an insertion in the priority queue, we simply append the
new priority to the list $L$ and, if necessary, update the value of
$\min(L)$.

To find the minimum element in the priority queue we simply return the
smaller of the two values stored at $\min(L)$ and $\min(T)$.

To delete the element $x$ from the priority queue given a pointer to
node the node $v$ with priority $x$, there are two cases we have to
consider.  

\noindent\textbf{Case 1:} (See \figref{queap-a}) 
The node $v$ is a node of $T$.  In this case, the deletion requires us
to update some $\min$ values at nodes of $T$.  In particular, If $u$
is the first node of the left wall on the path from $u$ to the root
then we must update the min values of all nodes on the path from $v$
to $u$ and on the path from $u$ to the leftmost node of $T$.  However,
each of these updates is easily done in constant time by traversing
upwards from $u$ to $v$ and then downwards from $v$ to the leftmost
node of $T$.  Finally, we may have to update the value $\min(T)$,
which can be done by looking at the second last node of the left roof
and its four children.

\begin{figure}
\centergraphics{queap-a}
\caption{Case~1 of deleting from a queap.  The tree nodes whose min values
 are affected are highlighted.}\figlabel{queap-a}
\end{figure}

\noindent\textbf{Case 2:} (See \figref{queap-b}) 
The node $v$ is a node of $L$.  In this case we
delete $v$ from $L$ and then we destroy the list $L$ and insert all
the priorities of $L$ into the tree $T$.  To do this, we first insert
the priorities of $L$ one at a time to the right of the rightmost
child of $T$.  Next, we have to update the $\min$ values in $T$.  Let
$T'$ be the subtree of $T$ obtained by taking all nodes on paths from
newly inserted nodes to the root of $T$.  The $\min$ values we need to
update are at the nodes of $T'$ plus the min values on the left wall
of $T$.  To update the $min$ values at the nodes of $T'$ we simply
perform a post-order traversal of $T'$.  Once this is done we can
update the min values on the left wall of $T$ by starting walking from
the root to the leftmost leaf.

\begin{figure}
\centergraphics{queap-b}
\caption{Case~2 of deleting from a queap.  The tree nodes whose min values
 are affected are highlighted.}\figlabel{queap-b}
\end{figure}

To analyze the running time of this priority queue implementation, we
use a credit scheme.  When a priority is inserted, it is given one
credit and it uses this credit later on to pay for the cost of moving
from the list $L$ to the tree $T$.  It is clear that the actual cost
of insertion is $O(1)$, and we give only one credit per insertion, so
the amortized cost of insertion is $O(1)$.  

To analyze the cost of deletion, we consider the two cases separately.
In Case~1 (the deleted node is in $T$), the cost is proportional
to the length of the path from the deleted node $v$ to the first node
$u$ on the left wall of $T$.  Suppose the length of this path is $l$.
Then the number of leaves in the subtree rooted at the leftmost child
of $u$ is at least $2^{l-1}$.  Furthermore, each of these leaves
corresponds to a priority that was inserted before the priority we are
deleting.  Therefore, $q(x)\ge 2^{l-1}$ and the cost of insertion is
$O(l)=O(\log q(x))$.

In Case~2, the actual cost of insertion is $O(|L|+\log(|L|+|T|))$.
However, during this insertion we have $L$ credits that can pay for
$O(L)$ work.  Also, $|T|$ is a lower bound on $q(x)$, i.e., all the
elements in $T$ were inserted before the element we are deleting.  Therefore,
if a credit pays for $c$ units of work, the amortized cost of the deletion is 
\begin{eqnarray*}
   \mbox{actual cost} - cL & = & O(|L|+\log(|L|+|T|)) - cL \\
	& = & O(|L|+\log(|T|)) -cL \\
	& = & O(\log|T|) \\
	& = & O(\log q(x)) \enspace,
\end{eqnarray*}
if $c$ is sufficiently large.

\begin{thm}
Queaps support insertion and finding the minimum element in $O(1)$ time
and deleting the element $x$ in $O(\log q(x))$ amortized time.
\end{thm}

\section{Self-Adjusting Lists}

Competitive analysis of the \textsc{Move-to-Front} (2-competitive) and
\textsc{Bit} ($<2$-competitive) list update rules. 

\section{Discussion and References}

Splay trees are due to Sleator and Tarjan \cite{st85}.  Since their
introduction, splay trees have been the topic of intensive study and
many conjectures about splay trees are still open today.  Queaps are
due to Iacono and Langerman \cite{il02}.

\bibliography{tds}
