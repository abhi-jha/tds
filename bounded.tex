%%%%%%%%%%%%%%%%%%%%%%%%%%%%%%%%%%%%%%%%%%%%%%%%%%%%%%%%%%%%%%%%%%%%%%%%
\chapter{Bounded Universes}\chaplabel{bounded}
\bibliographystyle{plain}

Until so far, most of the data structures described in this book have
been \emph{comparison based}.  Although they've been described as
storing real-valued keys, the only operations performed on keys have
been comparisons, so they could just as easily store keys from any
total order.  In this chapter we study the issue of what happens when
the keys are integers and we can do operations other than comparisons,
like addition, subtraction, integer division, and using the number to
index arrays.

More specifically, our keys come from the universe $U=\{0,\ldots,N-1\}$
and we assume a model that allows us to do most common mathematical
operations on integers and gives integer results by truncating (taking
the floor of) the results of operations when necessary.  The results
of operations can also be used as indices into arrays.

We are interested in data structures that support insertion, deletion,
membership queries and next element queries on keys from $U$.  The
meanings of insertion and deletion are obvious.  A membership query
takes as input a key $k$ and determines whether $k$ is currently
stored in the data structure.  An next element query takes a key $k$
and returns the smallest value $k'$ stored in the data structure such
that $k'\ge k$.

As a first attempt, we might consider using an array $A_1,\ldots,A_N$
of boolean values.  On creation, every array entry is initialized to
\texttt{false}.  When an element $k$ is inserted, we simply set $A_k$
to \texttt{true}.  Similarly, when an element $k$ is deleted we set
$A_k$ to \texttt{false}.  To do a membership query on the key $k$ we
just check the value of $A_k$.  Unfortunately, this is where this
approach breaks down.  The only way to do a next element query on key
$k$ is to check the values of $A_k$, $A_{k+1}$, $A_{k+2}$, and so on
until we find the first value $A_{k'}$ that is set to true.  Thus,
with this approach the first three operations can be done in constant
time, but the fourth (next element search) operation takes $\Omega(N)$ time
in the worst case.

%======================================================================
\section{Van Emde Boas Trees}

An interesting (and old) method of doing searches on bounded universes
was proposed by Peter~Van~Emde~Boas.  It is based on the recurrence
\begin{equation}
  T_N = \left\{\begin{array}{ll}
	O(1)+ T_{\sqrt{N}} & \mbox{if $N\ge 2$} \\
	O(1)               & \mbox{otherwise} 
	\end{array}\right. \eqlabel{rootish}
\end{equation}
which solves to $O(\log\log N)$ (because $N^{1/\log n}=2$).
Essentially, the above recurrence says that if we can, in constant
time, reduce the search space to the square root of its original size
and then recurse we will get a running time of $O(\log\log N)$.

The data structure we use to achieve this is called a VEB-tree.
Assume $N=2^{2^m}$ for some integer $m$.\footnote{We only make this
assumption so that $N^{1/2^i}$ is an integer for all integers $i<m$.
This allows us to avoid the need for floors, ceilings, and special
cases.  The modifications required when $N$ is not of this form should
be clear.}  The root of the VEB-tree for $U$ has $\sqrt{N}$ children
that are stored in an array.  Each child of the root is also a
VEB-tree, and the $i$th child corresponds to a VEB-tree for elements
$i\sqrt{N},\ldots,(i+1)\sqrt{N}-1$.  The root of the VEB-tree also
stores two integer values called min and max, which contain the
smallest and largest element currently contained in the tree.  If no
values are stored in the tree then min and max are set to some value
not in $U$, say $-1$.  A crucial point to remember about a VEB-tree is
that the min and max values at the root are only stored at the root.
Because of this, we can insert into an initially empty tree or delete
the last element from a tree in constant time.

The root of the VEB-tree also contains another VEB-tree---and this is
the truly clever part---for the universe
$\{0,\ldots,\sqrt{N}-1\}$. This auxilliary tree is used to keep track
of which children of the root contain data.  That is, the tree
contains the element $i$ if the $i$th child of the root contains some
key.

From this definition, it follows that the storage used by a VEB-tree is given by the recurrence
\[
  S_N = (\sqrt{N}+1)S_{\sqrt{N}} + \sqrt{N}
\]
which solves to $O(N)$.  (The extra +1 comes from the auxilliary VEB-tree.)

Although VEB-trees are not extremely complicated, their implementation
requires some care.  In the following three sections we sketch the
implementations of the algorithms for searching, inserting and
deleting in VEB-trees, and provide more precise pseudocode.  In the
pseudocode, $W$ is a VEB-tree node, $W[i]$ is the $i$th child of $W$,
$\child(k,W)$ is the index of the child of $W$ that stores the key
$k$, and $\aux(W)$ is the auxilliary VEB-tree stored at $W$.


\subsection{Searching}

To do a next element search for the key $k$ in a VEB-tree, we first
check if $k$ is less than the min value stored at the root.  If so,
then we simply report the min value.  Otherwise, we need to determine
which child of the root stores the key $k$.  This can be done in
constant time since it is the $i=\lfloor k/\sqrt{N}\rfloor$th child
and the children are stored in an array.  We then inspect the max
value for the $i$th child.  If it is larger than $k$, then we can be
sure that the key $k'$ we are looking for is in the subtree rooted at
the $i$th child and we recurse on the $i$th child.

Otherwise, the key we are looking for is contained in the $j$th child
of the root, where $j$ is the smallest value greater than $i$ such
that the $j$th child of the root contains some key.  In fact, the key
$k'$ we are looking for is the min value the $j$th child of the root.
Therefore, we can use the auxilliary tree at the root to find the
value of $j$ and then report the min value in constant time.

In both cases, the algorithm makes one recursive search call and does
$O(1)$ work.  The recursive search call is on a VEB-tree for a
universe of size $\sqrt{N}$.  Thus, the running time of the search
algorithm is given by the recurrence \eqref{rootish} and is
$O(\log\log N)$.

\noindent$\textsc{Successor}(k,W)$
\begin{algorithmic}[1]
\IF{$k<\min(W)$}
  \STATE{\textbf{output} $\min(W)$ \COMMENT{$k$ is smaller than every element}}
\ELSIF{$k>\max(W)$}
  \STATE{\textbf{output} $\infty$ \COMMENT{$k$ is larger than every element}}
\ENDIF
\STATE{$i\gets\child(k,W)$}
\IF{$\max(W[i])>k$}
  \STATE{$\textsc{Successor}(k,W[i])$} \COMMENT{$k'$ is stored in $W[i]$}
\ELSE
  \STATE{$j\gets\textsc{Successor}(i,aux(W))$} \COMMENT{$k'$ is in first
	non-empty sibling of $W[i]$} 
  \STATE{\textbf{output} $\min(W[j])$}
\ENDIF
\end{algorithmic}

\subsection{Inserting}

To insert the key $k$ into a VEB-tree we proceed as follows.  If the
root of $T$ is empty then we simply set the min and max values at the
root to be $k$.  Otherwise, we check if $k$ is less than (respectively
greater than) the min value (respectively max value) at the root.  If
so, we swap the values of $k$ and the min value (respectively max
value) before continuing.  Next, we find the child of the root that
should contain the key $k$ using the formula $i=\lfloor
k/\sqrt{N}\rfloor$.  If the $i$th child of the root contains no
elements then we insert $i$ into the root's auxilliary VEB-tree and
insert $k$ into the $i$th child of the root.  Otherwise (the $i$th
child already contains some element) we only insert $k$ into the $i$th
child of the root.

Observe that, because we explicitly check this condition, inserting
into an empty VEB-tree takes constant time.  This is very important,
because the VEB-tree algorithm may make two recursive insertion calls;
once to insert $k$ and once to insert $i$ into the auxilliarly
VEB-tree.  However, in this case, the recursive call to insert $k$
takes only constant time.  Thus, no matter what happens, the running
time of the insertion algorithm satisfies the ``rootish'' recurrence
\eqref{rootish} and therefore runs in $O(\log\log n)$ time.

\noindent$\textsc{Insert}(k,W)$
\begin{algorithmic}[1]
\IF{$\min(W)=\max(W)=-1$}
  \STATE{$\min(W)\gets\max(W)\gets k$ \COMMENT{tree is empty}} 
\ELSIF{$\min(W)=\max(W)$}
  \STATE{$\min(W)\gets\min\{k,\min(W)\}$} \COMMENT{tree contains 1 element}
  \STATE{$\max(W)\gets\max\{k,\max(W)\}$}
\ELSE
  \IF{$k<\min(W)$}
    \STATE{\textbf{swap} $k\leftrightarrow\min(W)$} \COMMENT{$k$ is 
	the new min, insert the old min}
  \ELSIF{$k>\max(W)$}
    \STATE{\textbf{swap} $k\leftrightarrow\max(W)$} \COMMENT{$k$ is 
	the new max, insert the old max}
  \ENDIF
  \STATE{$i\gets\child(k,W)$}
  \STATE{$\textsc{Insert}(k,W[i])$}
  \IF{$\max(W[i])=\min(W[i])=k$}
    \STATE{$\textsc{Insert}(i,\aux(W))$ \COMMENT{$W[i]$ just went from empty to non-empty}}
  \ENDIF
\ENDIF
\end{algorithmic}

\subsection{Deleting}  

Deleting the key $k$ from a VEB-tree is similar to insertion.  If the
tree contains only the element $k$, it is stored as the min and max
values of the root and we can delete it in constant time.  Otherwise,
if $k$ is equal to the min (respectively max) value at the root then
we swap $k$ and the min (respectively max) value at the root.  We then
recursively delete $k$ from the child of the root that contains it
and, if this child becomes empty we delete the child's index from the
auxiliary VEB-tree.

As with insertion, although there may be two recursive calls, only one
of them takes more than constant time.  Thus, the running time of the
deletion algorithm is given by recurrence \eqref{rootish} and runs
in $O(\log\log n)$.

\noindent$\textsc{Delete}(k,W)$
\begin{algorithmic}[1]
\IF{$\min(W)=\max(W)=k$}
  \STATE{$\min(W)\gets\max(W)\gets -1$ \COMMENT{ $k$ is the last element }}
\ELSIF{$\min(\aux(W))=-1$}
  \IF{$\min(W)=k$}
    \STATE{$\min(W)\gets\max(W)$}
  \ELSE
    \STATE{$\max(W)\gets\min(W)$}
  \ENDIF
\ELSIF{$k=\min(W)$}
  \STATE{$j\gets\min(\aux(W))$}
  \STATE{$\min(W)\gets\min(W[j])$}
  \STATE{$k\gets\min(W)$}
\ELSIF{$k=\max(W)$}
  \STATE{$j\gets\max(\aux(W))$}
  \STATE{$\max(W)\gets\max(W[j])$}
  \STATE{$k\gets\max(W)$}
\ENDIF
\STATE{$i\gets\child(k,W)$}
\STATE{$\textsc{Delete}(k,W[i])$}
\IF{$\min(W[i])=\max(W[i])=-1$}
  \STATE{$\textsc{Delete}(i,\aux(W))$}
\ENDIF
\end{algorithmic}


\begin{thm}
Van Emde Boas trees support insertion, deletion, and successor queries
for elements in the universe $U=\{0,\ldots,N-1\}$ in $O(\log\log N)$
time and require $O(N)$ storage.
\end{thm}

\section{Reducing Storage}

\section{Willard's X- and Y-Fast Trees}

Luc's notes go here.


%======================================================================
\section{Discussion and References}

The Van Emde Boas tree (VEB-tree) was introduced by van Emde Boas
\cite{veb74,vkz77}.  The description here was conveyed to us by
Michael Bender.  Since then, several variants have been introduced,
most with the goal of reducing the storage\ldots

\bibliography{tds}
